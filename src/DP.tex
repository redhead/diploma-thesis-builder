\documentclass[11pt,twoside,a4paper]{book}  
% definice dokumentu
\usepackage[czech, english]{babel}
\usepackage[T1]{fontenc} 				% pouzije EC fonty 
\usepackage[utf8]{inputenc} 			% utf8 kódování vstupu 
\usepackage[square, numbers]{natbib}	% sazba pouzite literatury
%\usepackage{indentfirst} 				% 1. odstavec jako v~cestine, pro práci v~aj možno zakomentovat
\usepackage{fancyhdr}					% tisk hlaviček a~patiček stránek
\usepackage{nomencl} 					% umožňuje snadno definovat zkratky a~jejich seznam

%%%%%%%%%%%%%%%%%%%%%%%%%%%%%%%%%%%%%%%%%%%%%%%%%%%%%%%%%%%%%%%
% informace o~práci
\newcommand\WorkTitle{Event Sourcing Design Pattern in a Java Enterprise Application}		% název
\newcommand\FirstandFamilyName{Radek Ježdík}															% autor
\newcommand\Supervisor{Ing. Ondřej Macek, Ph.D.}															% vedoucí

\newcommand\TypeOfWork{Master's Thesis}	% typ práce [Diplomová práce | Bakalářská práce | Bachelor's Project | Master's Thesis ]	

% Nastavte následují podle vašeho oboru a~programu (pomoc hledejte na http://www.fel.cvut.cz/cz/education/bk/prehled.html)								
\newcommand\StudProgram{Open Informatics}	% program
\newcommand\StudBranch{Software Engineering}           					% obor

%%%%%%%%%%%%%%%%%%%%%%%%%%%%%%%%%%%%%%%%%%%%%%%%%%%%%%%%%%%%%%%
% minimální importy
\usepackage{graphicx}					% pro vkládání obrázků
\usepackage{k336_thesis_macros} 		% specialni makra pro formatovani DP a~BP
\usepackage[
pdftitle={\WorkTitle},				% nastaví v~informacích o~pdf název
pdfauthor={\FirstandFamilyName},	% nastaví v~informacích o~pdf autora
colorlinks=true,					% před tiskem doporučujeme nastavit na false, aby odkazy a~url nebyly šedé při ČB tisku
breaklinks=true,
urlcolor=red,
citecolor=blue,
linkcolor=blue,
unicode=true,
]
{hyperref}								% pro zobrazování "prokliknutelných" linků 

% rozšiřující importy
\usepackage{listings} 			%slouží pro tisk zdrojových kódů se syntax higlighting
\usepackage{algorithmicx} 		%slouží pro zápis algoritmů
\usepackage{algpseudocode} 		%slouží pro výpis pseudokódu
\usepackage{enumitem}

%%%%%%%%%%%%%%%%%%%%%%%%%%%%%%%%%%%%%%%%%%%%%%%%%%%%%%%%%%%%%%%
% příkazy šablony
\makenomenclature								% při překladu zajistí vytvoření pracovního souboru se seznamem zkratek

\let\oldUrl\url									% url adresy budou zobrazeny: <url> 
\renewcommand\url[1]{<\texttt{\oldUrl{#1}}>}

%%%%%%%%%%%%%%%%%%%%%%%%%%%%%%%%%%%%%%%%%%%%%%%%%%%%%%%%%%%%%%%
% vaše vlastní příkazy
\newcommand*{\nomExpl}[2]{#2 (#1)\nomenclature{#1}{#2}} 	% usnadňuje zápis zkratek : Slova ke~Zkrácení (SZ)
\newcommand*{\nom}[2]{#1\nomenclature{#1}{#2}} 			% usnadňuje zápis zkratek : SZ


\usepackage{nameref}

\usepackage{hvfloat}

\usepackage{float}
\restylefloat{table}

\lstset{
	literate=%
		{á}{{\'a}}1
		{í}{{\'i}}1
		{é}{{\'e}}1
		{ý}{{\'y}}1
		{ú}{{\'u}}1
		{ó}{{\'o}}1
		{ě}{{\v{e}}}1
		{š}{{\v{s}}}1
		{č}{{\v{c}}}1
		{ř}{{\v{r}}}1
		{ž}{{\v{z}}}1
		{ď}{{\v{d}}}1
		{ť}{{\v{t}}}1
		{ň}{{\v{n}}}1
		{ů}{{\r{u}}}1
		{Á}{{\'A}}1
		{Í}{{\'I}}1
		{É}{{\'E}}1
		{Ý}{{\'Y}}1
		{Ú}{{\'U}}1
		{Ó}{{\'O}}1
		{Ě}{{\v{E}}}1
		{Š}{{\v{S}}}1
		{Č}{{\v{C}}}1
		{Ř}{{\v{R}}}1
		{Ž}{{\v{Z}}}1
		{Ď}{{\v{D}}}1
		{Ť}{{\v{T}}}1
		{Ň}{{\v{N}}}1
		{Ů}{{\r{U}}}1
}

\renewcommand{\lstlistingname}{Listing}

\usepackage{xcolor}
\definecolor{light-gray}{gray}{0.89}
\lstset{
	basicstyle=\ttfamily,
	backgroundcolor=\color{light-gray},
	caption={Listing},
	breakatwhitespace=false,
	breaklines=true
}

\usepackage{dirtree}

\usepackage{pdfpages}

\sloppy

%%%%%%%%%%%%%%%%%%%%%%%%%%%%%%%%%%%%%%%%%%%%%%%%%%%%%%%%%%%%%%%
% vlastní dokument
%%%%%%%%%%%%%%%%%%%%%%%%%%%%%%%%%%%%%%%%%%%%%%%%%%%%%%%%%%%%%%%
\begin{document}
	
	%%%%%%%%%%%%%%%%%%%%%%%%%% 
	% nastavení jazyka, kterým je práce psána
	\selectlanguage{english}	% podle jazyka práce nastavte na [czech | english]
	\translate				% nastaví české nebo anglické popisy (např. katedra -> department); viz k336_thesis_macros

	%%%%%%%%%%%%%%%%%%%%%%%%%%    
	% Poznamky ke~kompletaci prace
	% Nasledujici pasaz uzavrenou v~{} ve~sve praci samozrejme 
	% zakomentujte nebo odstrante. 
	% Ve~vysledne svazane praci bude nahrazena skutecnym 
	% oficialnim zadanim vasi prace.
%	{
%	\pagenumbering{roman} \cleardoublepage \thispagestyle{empty}
%	\chapter*{Zadání projektu}
%	Vytvořte nástroj pro tvorbu a~sdílení HTML5 prezentací na webu. Webový backend umožní  tvorbu a~editaci slidů a~jejich %sdílení. Pro prezentaci použijte vhodnou Java\-Scriptovou knihovnu, kterou rozšiřte o~možnost testování čtenářů prezentace. %Další požadavky na software sesbírejte na základě vytvořeného prototypu.\\\\
	
%	\noindent Očekávaný výstup aplikace:
%	\begin{enumerate}
%		\item Analýza, návrh a~implementace webové aplikace pro tvorbu, editaci a~sdílení prezentací
%		\item Rozšíření prezentační JS knihovny
%		\item Nasazení a~otestování aplikace
%	\end{enumerate}
%	\newpage
%	}

	%\pagenumbering{roman}
	%\includepdf{zadani.pdf}

	%%%%%%%%%%%%%%%%%%%%%%%%%%    
	% Titulni stranka / Title page 
	\coverpagestarts

	%%%%%%%%%%%%%%%%%%%%%%%%%%%    
	% Poděkovani / Acknowledgements 

	\acknowledgements
	\noindent
	I would like to give special thanks to my family and my girlfriend for their immense support and encouragement during my study at the university. Equally, I would like to thank my supervisor Ondřej Macek for his help and friendly communication, Sam for reviewing this text, and Zdeněk for his comments about CQRS and Event Sourcing.


	%%%%%%%%%%%%%%%%%%%%%%%%%%%   
	% Prohlášení / Declaration 

	\declaration{Prague, January 11\, 2015}


	%%%%%%%%%%%%%%%%%%%%%%%%%%%%    
	% Abstrakt / Abstract 
 
	\abstractpage

	This diploma thesis aims at the Event Sourcing design pattern and the closely associated Command Query Responsibility Segregation design pattern and their implementation in the Java programming language. The objective of this thesis was to study these design patterns and apply them to a real-world software project --- Integration Portal. The text introduces the patterns to the reader and explains their usage in Java. Furthermore, it describes the process of refactoring of the project and compares the new design with the traditional architecture found in most enterprise applications.

	% Prace v~cestine musi krome abstraktu v~anglictine obsahovat i
	% abstrakt v~cestine.
	\vglue60mm

	\noindent{\Huge \textbf{Abstrakt}}
	\vskip 2.75\baselineskip

	\noindent
	Tato práce se zaměřuje na návrhový vzor Event Sourcing a k němu přidružený Command Query Responsibility Segregation a jejich implementaci v programovacím jazyku Java. Cílem práce bylo nastudovat tyto návrhové vzory a použít je ve skutečném softwarovém projektu - Integračním portálu. V textu jsou tyto vzory čtenářům představeny spolu s popisem jejich použití v Javě. Dále text popisuje proces refaktorování projektu a porovnává tento nový design s tradiční architekturou nejčastěji používanou v podnikových aplikacích.

	%%%%%%%%%%%%%%%%%%%%%%%%%%    
	% obsahy a~seznamy
	\tableofcontents		% Obsah / Table of Contents 

	% pokud v~práci nejsou obrázky nebo tabulky - odstraňte jejich seznam
	\listoffigures			% Obsah / Table of Contents 
	% \listoftables			% Seznam tabulek / List of Tables

	%%%%%%%%%%%%%%%%%%%%%%%%%% 
	% začátek textu  
	\mainbodystarts




%%% DOC_CONTENT %%%



\bibliographystyle{plainnat}
%\bibliographystyle{plain}
%\bibliographystyle{csplainnat}
%{
%\def\CS{$\cal C\kern-0.1667em\lower.5ex\hbox{$\cal S$}\kern-0.075em $}
\bibliography{DP-ref}
%}


\nomenclature{CQRS}{Command Query Responsibility Segregation}
\nomenclature{ES}{Event sourcing}
\nomenclature{IoC}{Inversion of Control}
\nomenclature{ID}{Identifier}
\nomenclature{IDE}{Integrated Development Environment}
\nomenclature{JMS}{Java Messaging Services}
\nomenclature{JVM}{Java Virtual Machine}
\nomenclature{POJO}{Plain Old Java Object}
\nomenclature{JEE}{Java Enterprise Edition}
\nomenclature{ORM}{Object-Relational Mapping}
\nomenclature{DTO}{Data Transfert Object}
\nomenclature{DDD}{Domain-Driven Design}
\nomenclature{REST}{Representational State Transfer}
\nomenclature{CQS}{Command Query Separation}
\nomenclature{CRUD}{Creat, Read, Update, Delete}
\nomenclature{DAO}{Data Access Object}
\nomenclature{HTTP}{Hypertext Transfer Protocol}
\nomenclature{API}{Application programming interface}
\nomenclature{UI}{User interface}
\nomenclature{JPA}{Java Persistence API}
\nomenclature{JDBC}{Java Database Connectivity}
\nomenclature{URL}{Uniform Resource Locator}
\nomenclature{TDD}{Test-driven development}
\nomenclature{BDD}{Behavior Driver Development}
\nomenclature{OOP}{Object-oriented Programming}


%%%%%%%%%%%%%%%%%%%%%%%%%% 
% vše co následuje bude uvedeno v~přílohách
\appendix	

\printnomenclature

\chapter{Source code listings}\label{chap:ukazkykodu}


\begin{lstlisting}[caption={The \texttt{Label} aggregate source code},label={lst:labelAggregate},
language=java,
numbers=left,
breaklines=true]
@NoArgsConstructor
@Getter
public class Label extends
        AbstractAnnotatedAggregateRoot<LabelId> {

    @AggregateIdentifier
    private LabelId id;

    private UserId owner;

    private String name;

    private String color;

    private Set<NodeId> labeledNodes = new HashSet<NodeId>();

    public Label(LabelId id, String name, String color,
            UserId owner) {
        apply(new LabelCreatedEvent(id, name, color, owner));
    }

    public void edit(String name, String color) {
        if (this.name.equals(name) && this.color.equals(color)){
            return; // no change
        }
        apply(new LabelUpdatedEvent(id, name, color));
    }

    public void delete() {
        apply(new LabelDeletedEvent(id));
    }

    public void addToNode(NodeId nodeId) {
        if (labeledNodes.contains(nodeId)) {
            return; // no change
        }
        apply(new LabelAddedToNodeEvent(id, nodeId));
    }

    public void removeFromNode(NodeId nodeId) {
        if (!labeledNodes.contains(nodeId)) {
            return; // no change
        }
        apply(new LabelRemovedFromNodeEvent(id, nodeId));
    }

    @EventSourcingHandler
    public void handle(LabelCreatedEvent event) {
        id = event.getId();
        name = event.getName();
        color = event.getColor();
        owner = event.getOwner();
    }

    @EventSourcingHandler
    public void handle(LabelUpdatedEvent event) {
        name = event.getName();
        color = event.getColor();
    }

    @EventSourcingHandler
    public void handle(LabelDeletedEvent event) {
        markDeleted();
    }

    @EventSourcingHandler
    public void handle(LabelAddedToNodeEvent event) {
        labeledNodes.add(event.getNodeId());
    }

    @EventSourcingHandler
    public void handle(LabelRemovedFromNodeEvent event) {
        labeledNodes.remove(event.getNodeId());
    }
}
\end{lstlisting}

%\newpage

\begin{lstlisting}[caption={An example of a test specification using Spock.
The first method tests successful creation of a new label.
The second method tests that creating a duplicate label throws an exception.},label={lst:test},
language=java,
numbers=left,
breaklines=true]
public class CreateLabelIntegrationSpec
        extends AbstractIntegrationSpecification {

    def "should create a label"() {
        when:
            dispatch new CreateLabelCommand(LabelId.of("1"),
                "work", "red")

        then:
            def label = get(Label, "1")
            label.id == "1"
            label.name == "work"
            label.color == "red"
            label.owner.id == "1"
    }

    def "creating duplicate label results in error"() {
        given:
            dispatch new CreateLabelCommand(LabelId.of("1"),
                "work", "red")

        when:
            dispatch new CreateLabelCommand(LabelId.of("2"),
                "work", "red")

        then:
            thrown(AlreadyExistsException)
    }

}
\end{lstlisting}



\chapter{Contents of attached CD}

\dirtree{%
.1 src/\DTcomment{project source code}.
.1 text/.
.2 src/\DTcomment{thesis source code}.
.2 jezdirad-thesis-2015.pdf\DTcomment{thesis in PDF format}.
.1 instalacni\textunderscore{}manual.pdf\DTcomment{}.
.1 readme.txt\DTcomment{CD contents and important information}.
}



\end{document}

















